\documentclass[a4paper,12pt]{article}

\usepackage[margin=1in]{geometry}
\usepackage[noindentafter]{titlesec}
  \titleformat{\part}{\centering\bf}{}{0pt}{\MakeUppercase}{}
  \titleformat{\section}{\bf}{\thesection.}{1em}{}{}
 \titlespacing{\part}{0pt}{0pt}{0pt}
  \titlespacing{\section}{0pt}{0pt}{0pt}
\usepackage{parskip}
  \setlength{\parskip}{1em}
\usepackage{draftwatermark}
  \SetWatermarkScale{5}

\renewcommand{\labelenumi}{\thesection.\arabic{enumi}}
\renewcommand{\labelenumii}{\thesection.\arabic{enumi}.\arabic{enumii}}
\renewcommand{\labelenumiii}{(\roman{enumiii})}
\renewcommand{\labelenumiv}{(\Alph{enumiv})}

\begin{document}

\begin{center}
The \textit{Companies Act 2006}\par Articles of Association\par of\par RZZT CIC
\end{center}

\part{Interpretation}

\section{Defined terms}

The interpretation of these Articles is governed by the provisions set out in the Schedule at end of the Articles.

\part{Community Interest Company and Asset Lock}

\section{Community Interest Company}

The Company is to be a community interest company. % Mandatory provision

\section{Asset lock}

\begin{enumerate}
  \item The Company shall not transfer any of its assets other than for full consideration. % Mandatory provision
  \item Provided the conditions in Article 3.3 are satisfied, Article 3.1 shall not apply to % Mandatory provision
  \begin{enumerate}
    \renewcommand{\labelenumii}{(\alph{enumii})}
    \item the transfer of assets to any specified asset-locked body, or (with the consent of the Regulator) to any other asset-locked body, and % Mandatory provision
    \item the transfer of assets made for the benefit of the community other than by way of a transfer of assets into an asset-locked body. % Mandatory provision
  \end{enumerate}
  \item The conditions are that the transfer of assets must comply with any restrictions on the transfer of assets for less than full consideration which may be set out elsewhere in the Memorandum and Articles of the Company. % Mandatory provision
  \item If
  \begin{enumerate}
    \item the Company is wound up under the \textit{Insolvency Act 1986}, and
    \item all its liabilities have been satisfied
  \end{enumerate}
  any residual assets shall be given or transferred to the asset-locked body specified in Article 3.5 below.
  \item For the purposes of this Article 3, the following asset-locked body is specified as a potential recipient of the Company’s assets under Articles 3.2 and 3.4:\par Name: Public Software CIC\par Company Registration Number: 10026175\par Registered office: Fiander Tovell Llp Stag Gates House, 63/64 The Avenue, Southampton, United Kingdom, SO17 1XS
\end{enumerate}

\section{Not for profit}

The Company is not established or conducted for private gain: any surplus or assets are used principally for the benefit of the community.

\part{Objects, Powers and Limitation of Liability}

\section{Objects}

The objects of the Company are to carry on activities which benefit the community and in particular (without limitation) to

\begin{enumerate}
  \renewcommand{\labelenumi}{(\alph{enumi})}
  \item finance the development of free and open-source software that respects, protects or fulfills human rights and civil liberties, in particular economic rights, and the rights to privacy and freedom of expression,
  \item promote and provide free education about free and open-source software, and
  \item give out awards recognising excellence in the areas of free and open-source software and open culture.
\end{enumerate}

\section{Powers}

To further its objects the Company may do all such lawful things as may further the Company’s objects and, in particular, but, without limitation, may borrow or raise and secure the payment of money for any purpose including for the purposes of investment or of raising funds.

\section{Liability of members}


The liability of each member is limited to £1, being the amount that each member undertakes to contribute to the assets of the Company in the event of its being wound up while he or she is a member or within one year after he or she ceases to be a member, for

\begin{enumerate}
  \item payment of the Company’s debts and liabilities contracted before he or she ceases to be a member,
  \item payment of the costs, charges and expenses of winding up, and
  \item adjustment of the rights of the contributories among themselves.
\end{enumerate}

\part{Directors' Powers and Responsibilities}

\section{Directors' general authority}

Subject to the Articles, the Directors are responsible for the management of the Company’s business, for which purpose they may exercise all the powers of the Company.

\section{Members’ reserve power}

\begin{enumerate}
  \item The members may, by special resolution, direct the Directors to take, or refrain from taking, specific action.
  \item No such special resolution invalidates anything which the Directors have done before the passing of the resolution.
\end{enumerate}

\section{Chair}

The Directors may appoint one of their number to be the chair of the Directors for such term of office as they determine and may at any time remove him or her from office.

\section{Directors may delegate}

\begin{enumerate}
  \item Subject to the Articles, the Directors may delegate any of the powers which are conferred on them under the Articles
  \begin{enumerate}
    \item to such person or committee,
    \item by such means (including by power of attorney),
    \item to such an extent,
    \item in relation to such matters or territories, and
    \item on such terms and conditions
  \end{enumerate}
  as they think fit.
  \item If the Directors so specify, any such delegation may authorise further delegation of the Directors’ powers by any person to whom they are delegated.
  \item The Directors may revoke any delegation in whole or part, or alter its terms and conditions.
\end{enumerate}

\section{Commitees}

\begin{enumerate}
  \item Committees to which the Directors delegate any of their powers must follow procedures which are based as far as they are applicable on those provisions of the Articles which govern the taking of decisions by Directors.
  \item	The Directors may make rules of procedure for all or any committees, which prevail over rules derived from the Articles if they are not consistent with them.
\end{enumerate}

\part{Decision-Making by Directors}

\section{Directors to take decisions collectively}

Any decision of the Directors must be either a majority decision at a meeting or a decision taken in accordance with Article 19.

\section{Calling a Directors’ meeting}

\begin{enumerate}
  \item Two Directors may (and the Secretary, if any, must at the request of two Directors) call a Directors’ meeting.
  \item	A Directors’ meeting must be called by at least seven Clear Days’ notice unless either
    \begin{enumerate}
      \item all the Directors agree, or
      \item urgent circumstances require shorter notice.
    \end{enumerate}
  \item Notice of Directors’ meetings must be given to each Director.
  \item Every notice calling a Directors’ meeting must specify
    \begin{enumerate}
      \item	the place, day and time of the meeting, and
      \item if it is anticipated that Directors participating in the meeting will not be in the same place, how it is proposed that they should communicate with each other during the meeting.
    \end{enumerate}
  \item Notice of Directors’ meetings must be in Writing. % Changed from 'Notice of Directors’ meetings need not be in Writing.'
  \item	Notice of Directors’ meetings must be sent by Electronic Means to an Address provided by the Director for the purpose, but may also be given by additional means. % Changed from 'Notice of Directors’ meetings may be sent by Electronic Means to an Address provided by the Director for the purpose.'
\end{enumerate}

\section{Participation in Directors' meetings}

\begin{enumerate}
  \item Subject to the Articles, Directors participate in a Directors’ meeting, or part of a Directors’ meeting, when
  \begin{enumerate}
    \item the meeting has been called and takes place in accordance with the Articles, and
    \item they can each communicate to the others any information or opinions they have on any particular item of the business of the meeting.
  \end{enumerate}
  \item In determining whether Directors are participating in a Directors’ meeting, it is irrelevant where any Director is or how they communicate with each other.
  \item If all the Directors participating in a meeting are not in the same place, they may decide that the meeting is to be treated as taking place wherever any of them is.
\end{enumerate}

\section{Quorum for Directors’ meetings}

\begin{enumerate}
  \item At a Directors’ meeting, unless a quorum is participating, no proposal is to be voted on, except a proposal to call another meeting.
  \item The quorum for Directors’ meetings is two-thirds of the total number of Directors. % Changed from '1.1	The quorum for Directors’ meetings may be fixed from time to time by a decision of the Directors, but it must never be less than two, and unless otherwise fixed it is [two].'
  \item If the total number of Directors for the time being is less than the quorum required, the Directors must not take any decision other than a decision
  \begin{enumerate}
    \item to appoint further Directors, or
    \item to call a general meeting so as to enable the members to appoint further Directors.
  \end{enumerate}
\end{enumerate}

\section{Chairing of Directors' meetings}

The Chair, if any, or in his or her absence another Director nominated by the Directors present shall preside as chair of each Directors’ meeting.

\section{Decision-making at a meeting}

\begin{enumerate}
  \item Questions arising at a Directors’ meeting shall be decided by a majority of votes. % Mandatory provision
  \item In all proceedings of Directors each Director must not have more than one vote. % Mandatory provision
  \item In case of an equality of votes, the Chair shall have a second or casting vote.
\end{enumerate}

\section{Decisions without a meetings}

\begin{enumerate}
  \item The Directors may take a unanimous decision without a Directors’ meeting by indicating to each other by any means, including without limitation by Electronic Means, that they share a common view on a matter. Such a decision may, but need not, take the form of a resolution in Writing, copies of which have been signed by each Director or to which each Director has otherwise indicated agreement in Writing.
  \item A decision which is made in accordance with Article 19.1 shall be as valid and effectual as if it had been passed at a meeting duly convened and held, provided the following conditions are complied with:
  \begin{enumerate}
    \item approval from each Director must be received by one person being either such person as all the Directors have nominated in advance for that purpose or such other person as volunteers if necessary (`the Recipient'), which person may, for the avoidance of doubt, be one of the Directors,
    \item	following receipt of responses from all of the Directors, the Recipient must communicate to all of the Directors by any means whether the resolution has been formally approved by the Directors in accordance with this Article 19.2,
    \item	the date of the decision shall be the date of the communication from the Recipient confirming formal approval, and
    \item the Recipient must prepare a minute of the decision in accordance with Article 48.
  \end{enumerate}
\end{enumerate}

\section{Conflicts of interest}

\begin{enumerate}
  \item Whenever a Director finds himself or herself in a situation that is reasonably likely to give rise to a Conflict of Interest, he or she must declare his or her interest to the Directors unless, or except to the extent that, the other Directors are or ought reasonably to be aware of it already.
  \item If any question arises as to whether a Director has a Conflict of Interest, the question shall be decided by a majority decision of the other Directors.
  \item Whenever a matter is to be discussed at a meeting or decided in accordance with Article 19 and a Director has a Conflict of Interest in respect of that matter then, subject to Article 21, he or she must
  \begin{enumerate}
    \item	remain only for such part of the meeting as in the view of the other Directors is necessary to inform the debate,
    \item not be counted in the quorum for that part of the meeting, and
    \item withdraw during the vote and have no vote on the matter.
  \end{enumerate}
  \item	When a Director has a Conflict of Interest which he or she has declared to the Directors, he or she shall not be in breach of his or her duties to the Company by withholding confidential information from the Company if to disclose it would result in a breach of any other duty or obligation of confidence owed by him or her.
\end{enumerate}

\section{Directors’ power to authorise a conflict of interest}

\begin{enumerate}
  \item The Directors have power to authorise a Director to be in a position of Conflict of Interest provided that in relation to the decision to authorise a Conflict of Interest, the conflicted Director must comply with Article 20.3. % Slight format adjustment.
  \item In authorising a Conflict of Interest, the Directors can decide the manner in which the Conflict of Interest may be dealt with and, for the avoidance of doubt, they can decide that the Director with a Conflict of Interest can participate in a vote on the matter and can be counted in the quorum. % Slight format adjustment.
  \item A decision to authorise a Conflict of Interest can impose such terms as the Directors think fit and is subject always to their right to vary or terminate the authorisation. % Slight format adjustment.
  \item If a matter, or office, employment or position, has been authorised by the Directors in accordance with Article 21.1 then, even if he or she has been authorised to remain at the meeting by the other Directors, the Director may absent himself or herself from meetings of the Directors at which anything relating to that matter, or that office, employment or position, will or may be discussed.
  \item	A Director shall not be accountable to the Company for any benefit which he or she derives from any matter, or from any office, employment or position, which has been authorised by the Directors in accordance with Article 21.1 (subject to any limits or conditions to which such approval was subject).
\end{enumerate}

\section{Register of Directors’ interests}

The Directors shall cause a register of Directors’ interests to be kept. A Director must declare the nature and extent of any interest, direct or indirect, which he or she has in a proposed transaction or arrangement with the Company or in any transaction or arrangement entered into by the Company which has not previously been declared.

\part{Appointment and Retirement of Directors}

\section{Methods of appointing directors}

\begin{enumerate}
  \item Those persons notified to the Registrar of Companies as the first Directors of the Company shall be the first Directors.
  \item A person may not be appointed to be a Director if they have been convicted of, or are currently awaiting trial for, a criminal offence in any jurisdiction. % Inserted by request.
  \item Subject to these articles, any person who is willing to act as a Director, and is permitted by law to do so, may be appointed to be a Director
  \begin{enumerate}
    \renewcommand{\labelenumii}{(\alph{enumii})}
    \item by ordinary resolution, or
    \item	by a decision of the Directors,
  \end{enumerate}
  so long as the total number of Directors does not exceed three. % Inserted to limit the number of Directors
  \item In the event that ordinary resolutions are proposed that would, if passed, appoint more than three Directors, the Directors shall be appointed by full preferential instant run-off vote using a written open ballot, according to which the three most preferred persons shall be appointed as Directors. % Inserted to accommodate election of Directors where candidates exceeds three
  \item In any case where, as a result of death, the Company has no members and no Directors, the personal representatives of the last member to have died have the right, by notice in writing, to appoint a person to be a member.
  \item For the purposes of Article 23.5, where two or more members die in circumstances rendering it uncertain who was the last to die, a younger member is deemed to have survived an older member.
\end{enumerate}

\section{Termination of Director’s appointment}

A person ceases to be a Director as soon as

\begin{enumerate}
  \renewcommand{\labelenumi}{(\alph{enumi})}
  \item that person ceases to be a Director by virtue of any provision of the Companies Acts, or is prohibited from being a Director by law,
  \item a bankruptcy order is made against that person, or an order is made against that person in individual insolvency proceedings in a jurisdiction other than England and Wales or Northern Ireland which have an effect similar to that of bankruptcy,
  \item a composition is made with that person’s creditors generally in satisfaction of that person’s debts,
  \item notification is received by the Company from the Director that the Director is resigning from office, and such resignation has taken effect in accordance with its terms (but only if at least two Directors will remain in office when such resignation has taken effect),
  \item the Director fails to attend three consecutive meetings of the Directors and the Directors resolve that the Director be removed for this reason,
  \item at a general meeting of the Company, a resolution is passed that the Director be removed from office, provided the meeting has invited the views of the Director concerned and considered the matter in the light of such views, or
  \item the Director is convicted of a criminal offence in any jurisdiction. % Inserted by request
\end{enumerate}

\section{Directors’ remuneration}

\begin{enumerate}
  \item Directors may undertake any services for the Company that the Directors decide.
  \item Directors are entitled to such remuneration as the Directors determine
  \begin{enumerate}
    \renewcommand{\labelenumii}{(\alph{enumii})}
    \item for their services to the Company as Directors, and
    \item for any other service which they undertake for the Company.
  \end{enumerate}
  \item	Subject to the Articles, a Director’s remuneration may:
  \begin{enumerate}
    \item take any form, and
    \item include any arrangements in connection with the payment of a pension, allowance or gratuity, or any death, sickness or disability benefits, to or in respect of that director.
  \end{enumerate}
  \item Unless the Directors decide otherwise, Directors’ remuneration accrues from day to day.
  \item	Unless the Directors decide otherwise, Directors are not accountable to the Company for any remuneration which they receive as Directors or other officers or employees of the Company’s subsidiaries or of any other body corporate in which the Company is interested.
\end{enumerate}

\section{Directors’ expenses}

The Company may pay any reasonable expenses which the Directors properly incur in connection with their attendance at

\begin{enumerate}
  \renewcommand{\labelenumi}{(\alph{enumi})}
  \item meetings of Directors or committees of Directors,
  \item general meetings, or
  \item separate meetings of any class of members or of the holders of any debentures of the Company,
\end{enumerate}
  or otherwise in connection with the exercise of their powers and the discharge of their responsibilities in relation to the Company.

\part{Becoming and Ceasing to Be a Member}

\section{Becoming a member}

\begin{enumerate}
  \item The subscribers to the Memorandum are the first members of the Company. % Mandatory provision.
  \item Such other persons as are admitted to membership in accordance with the Articles shall be members of the Company. % Mandatory provision.
  \item No person shall be admitted a member of the Company unless he or she is approved by the Directors. % Mandatory provision.
  \item No person shall be admitted a member of the Company if they have been convicted of, or are currently awaiting trial for, a criminal offence in any jurisdiction. % Inserted by request.
  \item Every person who wishes to become a member shall deliver to the Company an application for membership in such form (and containing such information) as the Directors require and executed by him or her. % Mandatory provision.
\end{enumerate}

\section{Termination of membership}

\begin{enumerate}
  \item Membership is not transferable to anyone else. % Mandatory provision.
  \item Membership is terminated if% Mandatory provision.
  \begin{enumerate}
    \item the member dies or ceases to exist, % Mandatory provision.
    \item otherwise in accordance with the Articles, % Mandatory provision.
    \item the member is convicted of a criminal offence in any jurisdiction, or % Inserted by request.
    \item	at a meeting of the Directors at which at least half of the Directors are present, a resolution is passed resolving that the member be expelled on the ground that his or her continued membership is harmful to or is likely to become harmful to the interests of the Company. Such a resolution may not be passed unless the member has been given at least 14 Clear Days’ notice that the resolution is to be proposed, specifying the circumstances alleged to justify expulsion, and has been afforded a reasonable opportunity of being heard by or of making written representations to the Directors. A member expelled by such a resolution will nevertheless remain liable to pay to the Company any subscription or other sum owed by him or her.
  \end{enumerate}
\end{enumerate}

% \part{Organisation of General Meetings}

\end{document}
